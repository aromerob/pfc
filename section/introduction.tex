%This is the introduction of my Master Thesis
\section{Introduction}\label{introduction}
One of the biggest challenges in speech synthesis is the production of naturally sounding synthetic voices. This means that the resulting voice must be not only of high enough quality but also must be also that it must be able to capture the natural expressiveness imbued in human speech.\\
Speech synthesis is a field that has been seeing much more use in the last decade with the advent of human-machine interfaces, playing and integral role in them, so applications as telecommunication services, language education, help to people with disabilities, etc can be easily found. As such there have been constant studies on how to improve its quality, naturalness, expressiveness, etc.\\
Expressive speech synthesis is a sub-field of speech synthesis that has been drawing a lot of attention lately. Assign expressiveness (e.g. emotions or speaking styles) to the synthetic voices will lead to a much more natural voice, increasing the overall satisfaction of the end users of the interface.\\
Of the two main speech synthesis techniques (unit selection \cite{emospeech} and HMM based) HMM based synthesis has been used in this project due to its parametric nature is much more adaptable.
This project is focused in the production of emotional speech synthesis in Spanish language and it is focused in four emotions (anger, happiness, sadness and surprise) plus the neutral voice.\\
This is not the first attempt to do such a thing, it has been tried before and with succeed with the vocoder Straight (see \ref{straight}). So the goal is used the vocoder GlottHMM developed in Helsinki, that has been proof to be good in expressive speech recognition (see \ref{articulo jaime towards}) and compare it with Straight regarding the emotional speech synthesis. \\
One of the emotional speech synthesis a few years ago was to find a data base with enough data to train a robust model because emotional speech is not easy to find, so it has to be recorded in good conditions and for that money is needed.\\
The project is organized as follows. Information about the theory used in this project is presented in section \ref{background}. In section \ref{experiments} can be found the experiments that have been done and the steps to accomplish them. In section \ref{results} the results of the test performed with the synthesis samples achieved in the experiment can be found and in section \ref{conclusion} the conclusions of this project are exposed.