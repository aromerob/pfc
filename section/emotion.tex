\section{Emotion Analysis}\label{emotana}
One of the biggest problems found in research about speech is its variability. The intelligibility of the speech synthesizers is similar to the human one, but they do not have the variability of human speech which makes synthetic voice sound no natural.\\
The emotion is not a simple phenomenon, a lot of factors contribute to this.\\
Emotions are experienced when something unexpected happens and the emotional effects start to have control in those moments. So emotion can be also described as the interface of the organism with the outside world, pointing three main emotion functions:
\begin{itemize}
 \item Reflect the evaluation of the importance of a particular excitation in terms of the organism necessities, preferences, etc.
 \item Prepare physiologic and physically the organism for the appropriate action
 \item Notify the state of the organism and its intentions to other organisms that surround him. REPASAR ESTO CON EL ITS Y EL HIM, Y NO ME ACABA DE CONVENCER.
\end{itemize}
Emotion and mood are two different concepts, while emotions happen suddenly in response of a determined excitation and last seconds or minutes, the moods are more ambiguous in its nature and can last hours or days.\\
A lot of the words used to define emotions and its effects are necessary diffuse and are not clearly defined. This can be explained due to the difficulty for expressing with words abstract concepts that can not be quantified. For that reason, to describe the characteristic  of the emotions a group of emotive words are used, but most of them are selected for personal choice.\\
The first researches about how the emotions affect to the behavior and the language of the animals were briefly described by Darwin in his book \textit{The Effect of Emotion in Man and Animals}, publish in 1872. Lately, the effects of the emotions in speech have been studied by acoustic researchers that have analyzed the speech signal, by linguist, that have studied the lexical and prosody effects, and by psychologist . Thanks to them a lot of components present in emotions have been identified. The more important are: pitch, duration and voice quality.\\
The pitch (F0) is the fundamental frequency at which the vocal cords vibrates. The characteristic of the pitch are some of the main source of information about emotions. For example:
\begin{itemize}
 \item The average value of F0 express the level of excitation of the speaker, so a high average of F0 means a higher level of excitement
 \item The range of F0 is the distance between the maximum and minimum value of the F0. It also reflects the level of excitation of the speaker
 \item Fluctuations in F0, defined as the speed of the fluctuation between high and low values and if they are blunt or soft
\end{itemize}
The duration is the component of prosody described by the speed of the speech and the situation of the accents, and which effects are the rhythm and the speed. Emotions can be distinguish for some features as:
\begin{itemize}
	\item Speech speed: usually an excited speaker will reduce the duration of syllables
	\item Number of pauses and its duration: an excited speaker will tend to speak faster, with less and shorter pauses, while a depressed speaker will speak slower and with bigger pauses
	\item Quotient between speak and pauses time
\end{itemize}
The quality of the speech can be distinguish by:
\begin{itemize}
	\item Intensity: is related with the perception of the volume
	\item Voice irregularities: the speech jitter reflects the fluctuations of F0 of a glottal pulse to the other (like in angry emotion) or the disappearance of speech in some emotions (like sadness)
	\item The quotient between high and low frequencies: a big amount of energy in high frequencies is associated with the angry emotion, while low amount of energy is related with sadness
	\item Breathiness (parece que asi no existe y no estoy seguro a que se refiere) and larynx effects reflects the characteristics of the vocal tract that are related with the customization of each voice.
\end{itemize}
Joel Davitz and klaus Scherer classified the emotions and its effects using three edges of the semantic field:
\begin{itemize}
	\item Power or Strength: corresponds to the attention or rejection, differentiating between emotions started by a subject to the ones that appear of the environment
	\item Pleasure or evaluation: according to the pleasant or unpleasant of the emotion
	\item Activity: presence or absence of energy or tension
\end{itemize}
Thank to some research it has been discovered that emotions with a same lever of activity are easier to confuse that the ones that have a similar level of strength or pleasure. So the activity is more related with simple hearing variables as tone or intensity.\\
Some researchers have divided the emotions into two groups, so an emotion can be:
\begin{itemize}
	\item Active: which qualities are a low speech speed, low volume, low tone and a more resonant timbre
	\item Pasive: which qualities are a high speech speed, high volume, high tone and a "on" timbre
\end{itemize}
Puedo poner algo general de las emociones sin especificar freq y eso de lo del ingles..pero es arriesgarse\\
More information about emotions like biological reasons can be found in \ref{http://dspace.universia.net/bitstream/2024/195/1/Trabajo+imprimible.pdf}
